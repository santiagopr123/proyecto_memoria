\documentclass{article}
\usepackage[utf8]{inputenc}
\usepackage[spanish]{babel}
\usepackage{listings}
\usepackage{graphicx}
\graphicspath{ {images/} }
\usepackage{cite}

\begin{document}

\begin{titlepage}
    \begin{center}
        \vspace*{1cm}
            
        \Huge
        \textbf{trabajo de investigacion}
        \vspace{0.5cm}
        \LARGE
        memoria en las computadoras
            
        \vspace{1.5cm}
            
        \textbf{Santiago Pereira Ramirez}
            
        \vfill
            
        \vspace{0.8cm}
            
        \Large
        Despartamento de Ingeniería Electrónica y Telecomunicaciones\\
        Universidad de Antioquia\\
        Medellín\\
        Septiembre de 2020
            
    \end{center}
\end{titlepage}

\tableofcontents

\section{Sección introductoria}
durante mucho tiempo se ha pensado que programar es mucho mas que  escribir codigo, en gran parte no deberia ser asi,ni mucho menos pensado.en la programacion para programar en un buen sentido de la palabra se debe de tener aspectos tales como la comprension y calidad del algoritmo que vamos a implementar,tener diferentes capacidades de comunicacion 
y formas de trabajar en equipo,asi como tener la disciplina y el empeño para realizar los diferentes trabajos asignados .Pero algo que es de sumamente importancia 
no solo al programar sino tambien en cualquier otro trabajo en el mundo el cual es conocer la herramien de trabajo y como lo que haremos influir en los dintintos 
tipos de dispositivos, a continuacion mostraremos aspectos muy importantes sobre la memoria, algo mas alla de guardar.

\section{Sección de contenido} \label{contenido}
    \subsection{Qué es la memoria de un computador?}
    \subsection{Tipos de memoria}
    \subsection{¿Como se debe de gestionar una memoria en un computador?}
    \subsection{Qué hace que una memoria sea más rápida que otra y porque esto es importante?}
    
\cite{dirac}

\begin{lstlisting}

\end{lstlisting}

A continuación se presenta el logo de C++ Figura (\ref{fig:cpplogo})

\begin{figure}[h]
\includegraphics[width=4cm]{cpplogo.png}
\centering
\caption{Logo de C++}
\label{fig:cpplogo}
\end{figure}

(\ref{contenido})

\section{Conclusión} \label{conclulsion}

\bibliographystyle{IEEEtran}
\bibliography{references}

\end{document}
